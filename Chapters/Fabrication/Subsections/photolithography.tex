\subsection{Photolithography}
\label{sec:fab:photolithography}

% \hl{9.375\% Briefly describe the processes carried out in steps 2 and 10. Comment on the exposure parameters and choice of developer. [250 words max + Figures]}

For the photolithography steps, the substrate was placed on a suitably sized vacuum chuck in a resist spinner. The vacuum was checked before spinning to ensure the substrate did not come off at high spin speeds. The resist used was Microposit S1818 which was spun at 4000rpm for 30 seconds. According to the S1800 series datasheet \cite{s18}, this would result in the desired resist thickness of $1.8\mu m$. The S1818 was filtered as it was deposited and it was made sure the the majority of the substrate was covered prior to spinning

After spinning the back of the sample was cleaned with cotton bud soaked with acetone. This removed any unwanted photoresist and stopped the back of the sample contaminating or sticking to the hotplate.

After spinning the substrate was baked on a hotplate at $115\degree C$ for 120 seconds. This hardened the photoresist enough so that it would not stick to the photomask or that the resist structure would not collapse and distort after developing.

The SUSS MicroTec MJB4 mask aligner was used to align the masks on the substrate with a hard contact which results in more precise developing when compared to a soft contact%The resist was exposed for \hl{XXXX} seconds.

The resist was then developed in a 1:1 H$_2$O:Microposit developer concentrate for 75 seconds. After being rinsed in RO water and dried with the N2 gun, the substrate was again ashed in an oxygen plasma for 3 minutes at 150W to remove any residual resist.
