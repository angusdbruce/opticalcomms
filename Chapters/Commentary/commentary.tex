\section{Commentary}
\label{sec:comment}

% \hl{15.625\%
%
% Highlight any process steps that you think may need more attention should you repeat the fabrication process.
% How would you redesign the process sequence if the scribe and break at process number 8 could be moved to the end? Comment upon how the LED operation varies with mesa diameter. Suggest how you may make the LED operate more efficiently. [300 words max + Figures].}

If the fabrication process were to be repeated in future, the etch time would be reduced since it etched $~67\%$ deeper than intended. The time would initially be halved then the depth measured before continuing the etching in 5 second periods. If the photomasks were to be redesigned at any point, careful consideration should be given to the alignment markers since they did not align.

If the scribe and break process was moved to the end, there it would be possible to us the CTLM to characterise a much larger number of LEDs with the same amount of measurements. Approximately $25\%$ of the current design is given over to testing whereas the same area could be used on a much larger wafer and drastically increase the yield.

The larger mesa diameters used by other students resulted in both more optically powerful LEDs and a larger current draw.
